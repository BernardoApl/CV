%%%%%%%%%%%%%%%%%%%%%%%%%%%%
%%%% CV - Bernardo Lopes %%%%%
%%%%%%%%%%%%%%%%%%%%%%%%%%%%

\documentclass[10pt,a4paper,ragged2e,withhyper,hidelinks]{altacv}

%Layout
\geometry{left=1.5cm,right=1.5cm,top=1.5cm,bottom=1.5cm,columnsep=1.2cm,paperheight=50cm}

%Packages
\usepackage{paracol}
\usepackage{gensymb}

% Remove automatic hyphenation
\usepackage[none]{hyphenat}
\let\raggedright\relax

% Change the font if you want to, depending on whether
% you're using pdflatex or xelatex/lualatex
\ifxetexorluatex
  % If using xelatex or lualatex:
  \setmainfont{Lato}
\else
  % If using pdflatex:
  \usepackage[default]{lato}
\fi

%Colors
\definecolor{SlateGrey}{HTML}{2E2E2E}
\definecolor{LightGrey}{HTML}{666666}
\definecolor{GoldenEarth}{HTML}{E7D192}
\definecolor{VividPurple}{HTML}{3E0097}

\colorlet{name}{black}
\colorlet{tagline}{VividPurple}
\colorlet{heading}{VividPurple}
\colorlet{headingrule}{VividPurple}
\colorlet{subheading}{VividPurple}
\colorlet{accent}{VividPurple}
\colorlet{emphasis}{black}
\colorlet{sub}{SlateGrey}
\colorlet{body}{black}

\renewcommand{\itemmarker}{{\small\textbullet}}
\renewcommand{\ratingmarker}{\faCircle}

\newcommand{\cvsubevent}[4]{%
    \sbox0{{\textbf{\color{accent}#2}}}%
    \sbox1{{\small\color{sub}\makebox[0.26\linewidth][l]{\cvDateMarker~#3}}}%
    \noindent%
    \makebox[\linewidth][s]{%
        \usebox0%
        \hfill%
        \textcolor{sub!30}{\hdashrule[0.5ex]{\dimexpr\linewidth-\wd0-\wd1-2\intextsep\relax}{0.6pt}{0.5ex}}%
        \hfill%
        \usebox1%
    }%
  \medskip\normalsize
}

\newcommand{\cveventinv}[4]{%
  {\large\color{emphasis}#1}
  \smallskip\normalsize
  \ifstrequal{#2}{}{}{
  \textbf{\color{accent}#2}
  \smallskip}
  \ifstrequal{#4}{}{}{{\small\hfill\color{emphasis}\makebox[0.1\linewidth][l]{\faMapMarker~#4}}}%
  \ifstrequal{#3}{}{}{{\small\hfill\color{emphasis}\makebox[0.26\linewidth][l]{\faCalendar~#3}}}\par
  \medskip\normalsize
}

\newcommand{\cvproject}[3]{%
    \sbox0{{\textbf{\small\color{accent}{#1}}}}%
    \sbox1{{\small\color{emphasis}#3\hspace{0.5em}\makebox[0.33\linewidth][l]{#2}}}%
    \noindent%
    \makebox[\linewidth][s]{%
        \usebox0%
        \hfill%        \textcolor{sub!30}{\hdashrule[0.5ex]{\dimexpr\linewidth-\wd0-\wd1-2\intextsep\relax}{0.6pt}{0.5ex}}%
        \hfill%
        \usebox1%
    }%
  \medskip\normalsize
}

%

\begin{document}
\name{Bernardo Lopes}
\tagline{Platform, Site Reliability, e Engenharia de Software}
%\photo{2.5cm}{}
\personalinfo{%
 \location{Betim, Brasil}
\email{b.lopes.software.com}
\homepage{} % fazer depois
\linkedin{bernardo-lopes-3500b92b6/}
\github{BernardoAPL}

}

\begin{fullwidth}
  \makecvheader
\end{fullwidth}


%% Set the left/right column width ratio to 6:4.
\columnratio{0.6}

% Start a 2-column paracol. Both the left and right columns will automatically
% break across pages if things get too long.
\begin{paracol}{2}

  %%%%%%%%%%%%%%%%%%%%%%%%%%%%%%% About %%%%%%%%%%%%%%%%%%%%%%%%%%%%%%%

  \cvsection{Sobre}

  \small{Minha especialidade está em \textbf{Site Reliability Engineering - SRE}, fortalecida por um foco em \textbf{Software} e \textbf{Software Management}. Minha paixão está em construir plataformas robustas para \textbf{clientes}, criar \textbf{infraestrutura como código} de maneira sustentável e arquitetar sistemas de forma simples e direta. Minha experiência como desenvolvedor, com foco em mini projetos em \textbf{java}, \textbf{React} e \textbf{FrontEnd(HTML,CSS,JS}, me permite criar plataformas que não são apenas confiáveis e fáceis de manter, mas também prazerosas de usar.}


  %\textcolor{emphasis}{Certifications:}\smallskip
  %\cvachievement{\faTrophy}{Fantastic Achievement}{and some details about it}

  \switchcolumn

  %%%%%%%%%%%%%%%%%%%%%%%%%%%%%%% Education %%%%%%%%%%%%%%%%%%%%%%%%%%%%%%%

  \cvsection{Educação}

  \cvevent{E.S.\ Engenharia de Software}{PUC Minas - Coração Eucaristico}{Jan 2024 -- Atual}{}

  \begin{itemize}
    \setlength{\itemindent}{0em}
    \item   \small{Gerência de Projetos}
    \item   \small{Produção e Desenvolvimento de Software}
  \end{itemize}
  \cvevent{Técnico em Automação Industrial}{SENAI Maria Madalena Nogueira}{2021 -- 2023}{Betim, MG}

\begin{itemize}
  \setlength{\itemindent}{0em}
  \item \small{Formação técnica com foco em eletrônica, sensores, CLPs e sistemas automatizados.}
  \item \small{Programação de controladores lógicos (CLP) e integração de sistemas industriais.}
  \item \small{Projetos práticos envolvendo circuitos elétricos, pneumática, robótica e automação de processos.}
\end{itemize}


\end{paracol}

\medskip
\smallskip

%%%%%%%%%%%%%%%%%%%%%%%%%%%%%%% Skills %%%%%%%%%%%%%%%%%%%%%%%%%%%%%%%

\cvsection{Skills}

\small{
  \textcolor{emphasis}{Languages: }
  \smallskip
  \cvtag{Java}
  \cvtag{TypeScript}
  \cvtag{JavaScript}
  \cvtag{C++}
  \cvtag{HTML}
  \cvtag{CSS}
}

\smallskip

\small{
  \textcolor{emphasis}{Frameworks: }
  \smallskip
  \cvtag{React}
  \cvtag{Spring Boot}
}

\smallskip

\small{
  \textcolor{emphasis}{Infrastructure \& Automation: }
  \smallskip
  \cvtag{PostgreSQL}
  \cvtag{MySQL}
  \cvtag{Firebase}
  \cvtag{n8n}
}

\smallskip

\small{
  \textcolor{emphasis}{DevOps: }
  \smallskip
  \cvtag{Git}
  \cvtag{GitHub Actions}
}

\smallskip

\small{
  \textcolor{emphasis}{Tools: }
  \smallskip
  \cvtag{VS Code}
  \cvtag{IntelliJ}
  \cvtag{Figma}
}


\medskip
\smallskip

%%%%%%%%%%%%%%%%%%%%%%%%%%%%%%% Experience %%%%%%%%%%%%%%%%%%%%%%%%%%%%%%%

\cvsection{Experiência}

\medskip

\cveventinv{Prefeitura Municipal de Betim}{Ouvidoria - Estatística}{Março 2024 -- Julho 2025}{Betim, MG}

\cvsubevent{}{Projetos e Atividades Desenvolvidas}{}{}

\begin{itemize}
  \item \small{Desenvolvimento de dashboards avançados em \textbf{Power BI} integrados com múltiplas fontes de dados.}
  \item \small{Automação de relatórios por meio da integração entre \textbf{Excel} e diversas planilhas corporativas.}
  \item \small{Modelagem e análise estatística para geração de relatórios estratégicos utilizados pela gestão municipal.}
  \item \small{Criação de relatórios analíticos e cálculos automatizados para apoio à tomada de decisão.}
\end{itemize}

\medskip

\cveventinv{Prefeitura Municipal de Betim}{Estagiário de Tecnologia}{Agosto 2025 -- Atual}{Betim, MG}

\cvsubevent{}{Atividades e Responsabilidades}{}{}

\begin{itemize}
  \item \small{Prestação de \textbf{suporte técnico} a usuários do sistema, garantindo estabilidade e rápida resolução de ocorrências.}
  \item \small{Desenvolvimento e programação de \textbf{soluções internas} para corrigir problemas recorrentes e otimizar fluxos de trabalho.}
  \item \small{Apoio na manutenção e melhoria contínua dos sistemas utilizados pelos setores internos da prefeitura.}
  \item \small{Colaboração com equipes operacionais para automatizar tarefas e reduzir falhas manuais.}
\end{itemize}

\medskip


\medskip

\smallskip
%\newpage

%%%%%%%%%%%%%%%%%%%%%%%%%%%%%%% Projects %%%%%%%%%%%%%%%%%%%%%%%%%%%%%%%

\cvsection{Projetos Principais}

\cvproject{CarExpress}{\href{https://github.com/BernardoApl/CarExpress}{github.com/BernardoApl/CarExpress}}{\faGithub}
\begin{itemize}
  \item \small{Sistema completo de gerenciamento de veículos, com funcionalidades de cadastro, consulta, atualização, remoção, cálculo de distância entre cidades, backup automático e restauração.}
  \item \small{Desenvolvido como aplicação CLI em \textbf{C++}, utilizando programação orientada a objetos, arquivos binários \texttt{.dat}, tratamento de exceções, vetores e structs.}
  \item \small{Tecnologias: C++, POO, Arquivos binários, CLI}
\end{itemize}

\medskip

\cvproject{Sistema Pesque e Pague}{\href{https://github.com/ICEI-PUC-Minas-PMGES-TI/pmg-es-2025-2-ti2-3740100-pesque-e-pague}{github.com/ICEI-PUC-Minas.../pesque-e-pague}}{\faGithub}
\begin{itemize}
  \item \small{Plataforma completa para gestão de pesque e pagues, integrando estoque, vendas, reservas, aluguel de equipamentos e relatórios financeiros em um único sistema.}
  \item \small{Projetado para substituir processos manuais, aumentar eficiência operacional e melhorar a experiência do cliente.}
  \item \small{Tecnologias: TypeScript, React, Java, Spring Boot, PostgreSQL}
\end{itemize}

\medskip

\cvproject{Cálculo de Folha de Pagamento}{\href{https://github.com/pm-puc-minas/calculo-folha-pagamento-lab-1-grupo-1}{github.com/pm-puc-minas/calculo-folha-pagamento}}{\faGithub}
\begin{itemize}
  \item \small{Sistema de folha de pagamento que calcula salários, adicionais, benefícios, descontos legais (INSS, FGTS, IRRF) e gera relatório completo da folha para os usuários.}
  \item \small{Inclui frontend em \textbf{TypeScript + React} e backend em \textbf{Java Spring Boot}, com testes unitários e APIs REST para gerenciamento de funcionários.}
  \item \small{Tecnologias: Java Spring Boot, React, TypeScript, CSS3, PostgreSQL}
\end{itemize}

\medskip


\end{document}
